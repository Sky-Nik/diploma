\section{Результати}

Для кожного алгоритму і кожного розміру задачі було проведено $5$ запусків (із різними матрицями), у таблиці і на графіку наведені середні значення та середньоквадратичні відхилення.

\begin{figure}[H]
    \centering
    \includegraphics[width=.75\textwidth]{img/2/time.png}
\end{figure}

Та сама інформація у табличці, для зручності:

\begin{table}[H]
	\centering
	\begin{tabular}{|c||c|c|c|c|}\hline
		Розмір задачі & 100 & 200 & 500 & 1000 \\ \hline \hline
		Корпелевич & 0.4350947380065918 $\pm$ 0.1913426633302341 & 0.90950608253479 $\pm$ 0.3372043295482586 & 2.248037815093994 $\pm$ 0.6865537238656925 & 7.862243127822876 $\pm$ 0.8350072687434266 \\ \hline
		Tseng & 0.40647144317626954 $\pm$ 0.15133783648646926 & 0.8260695457458496 $\pm$ 0.1772062299787947 & 1.8909038066864015 $\pm$ 0.4069383251247664 & 6.568820381164551 $\pm$ 0.6888329537322256 \\ \hline
		Кеш. Tseng & 0.38817553520202636 $\pm$ 0.11465933017314325 & 0.7749995708465576 $\pm$ 0.1708256909551331 & 1.591792058944702 $\pm$ 0.42990579543045015 & 5.008282089233399 $\pm$ 0.6336524653402142 \\ \hline
		Попов & 0.42557611465454104 $\pm$ 0.17081812322670448 & 1.0770569324493409 $\pm$ 0.2855756813393744 & 2.7644895553588866 $\pm$ 0.7016420097186868 & 7.4928144931793215 $\pm$ 0.46711583329394046 \\ \hline
		Кеш. Попов & 0.36380467414855955 $\pm$ 0.17619206505491763 & 0.6023744583129883 $\pm$ 0.14104575916143772 & 2.428965997695923 $\pm$ 0.6790339589256428 & 6.353133010864258 $\pm$ 0.4615922554022695 \\ \hline
	\end{tabular}
	\caption{Час виконання, секунд}
\end{table}


У цій задачі основна складність все ще у обчисленні оператора $A$, %($O(m^2)$)
хоча обчислення проекції вже більш складне, %($O(m \log m)$)
тому алгоритм Tseng'a має певну перевагу над алгоритмом Попова, який у свою чергу випереджає алгоритм Корпелевич. Щодо кількості ітерацій то усі три алгоритми демонструють практично ідентичні результати.

\begin{table}[H]
	\centering
	\begin{tabular}{|c||c|c|c|c|}\hline
		Розмір задачі & 100 & 200 & 500 & 1000 \\ \hline \hline
		Корпелевич & 1000.00 $\pm$ 0.00 & 1000.00 $\pm$ 0.00 & 1000.00 $\pm$ 0.00 & 1000.00 $\pm$ 0.00 \\ \hline
		Tseng & 1000.00 $\pm$ 0.00 & 1000.00 $\pm$ 0.00 & 1000.00 $\pm$ 0.00 & 1000.00 $\pm$ 0.00 \\ \hline
		Кеш. Tseng & 1000.00 $\pm$ 0.00 & 1000.00 $\pm$ 0.00 & 1000.00 $\pm$ 0.00 & 1000.00 $\pm$ 0.00 \\ \hline
		Попов & 1000.00 $\pm$ 0.00 & 1000.00 $\pm$ 0.00 & 1000.00 $\pm$ 0.00 & 1000.00 $\pm$ 0.00 \\ \hline
		Кеш. Попов & 1000.00 $\pm$ 0.00 & 1000.00 $\pm$ 0.00 & 1000.00 $\pm$ 0.00 & 1000.00 $\pm$ 0.00 \\ \hline
		Маліцький Tam & 1000.00 $\pm$ 0.00 & 1000.00 $\pm$ 0.00 & 1000.00 $\pm$ 0.00 & 1000.00 $\pm$ 0.00 \\ \hline
		Кеш. Маліцький Tam & 1000.00 $\pm$ 0.00 & 1000.00 $\pm$ 0.00 & 1000.00 $\pm$ 0.00 & 1000.00 $\pm$ 0.00 \\ \hline
	\end{tabular}
	\caption{Число ітерацій}
\end{table}


Знову ж таки, кешування дає перевагу на великих задачах, хоча вона вже не у 1.5--2 рази. \medskip

\begin{remark}
    Можна кешувати якийсь із матрчиних розкладів $M$ для пришвидшення множення $M x$. 
\end{remark}

\begin{remark}
    Наша реалізація приблизно у 2000 разів швидша за результати наведені у статті \href{https://arxiv.org/abs/1502.04968v1}{[Yura Malitsky, 2015]}. Краща машина, зміна мови програмування та алгоритму проектування на явний є причинами пришвидшення приблизно у співвідношенні $1 : 1 : 2$. 
\end{remark}