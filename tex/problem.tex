Для порівняння алгоритмів нам знадобляться тестові задачі різної складності та різних розмірів. У якості таких задачі розглянемо:

\section{Перша задача}

Класичний приклад. Допустимою множиною є увесь простір: $C = \RR^m$, а $F(x) = Ax$, де $A$ --- квадратна $m \times m$ матриця, елементи якої визначаються наступним чином:
\begin{equation}
    a_{i,j} = \begin{cases}
        -1, & j = m - 1 - i > i, \\
        1, & j = m - 1 - i < i, \\
        0, & \text{інакше}.
    \end{cases}
\end{equation}

\begin{remark}
    Тут і надалі нуметрація рядків/стовпчиків матриць, а також елементів масивів починається з нуля. Якщо у вашій мові програмування нумерація починається з одиниці то у виразах вище замість $m - 1$ має бути $m + 1$.
\end{remark}

Це визначає матрицю, чия бічна діагональ складається з половини одиниць і половини мінус одиниць, а решта елементів якої нульові. Для наглядності наведемо декілька преших матриць, для $m = 2, 4, 6$:
\begin{equation}
    \begin{pmatrix}
        0 & -1 \\
        1 & 0
    \end{pmatrix}
    \qquad
    \begin{pmatrix}
        0 & 0 & 0 & -1 \\
        0 & 0 & -1 & 0 \\
        0 & 1 & 0 & 0 \\
        1 & 0 & 0 & 0
    \end{pmatrix}
    \qquad
    \begin{pmatrix}
        0 & 0 & 0 & 0 & 0 & -1 \\
        0 & 0 & 0 & 0 & -1 & 0 \\
        0 & 0 & 0 & -1 & 0 & 0 \\
        0 & 0 & 1 & 0 & 0 & 0 \\
        0 & 1 & 0 & 0 & 0 & 0 \\
        1 & 0 & 0 & 0 & 0 & 0
    \end{pmatrix}
\end{equation}

Для парних $m$ нульовий вектор є розв'язком відповідної варіаційної нерівності \eqref{eq:variational-inequality}.

\begin{remark}
    Для цієї задачі $P_C = \text{Id}$, а тому алгоритми Корпелевич і Tseng'a еквівалентні. Втім, некешована версія алгоритму Tseng'a буде працювати повільніше, що ми скоро і побачимо.
\end{remark}


\section{Друга задача}

Візьмемо $F(x) = M x + q$, де матриця $M$ генерується наступний чином:
\begin{equation}
    M = A A^\intercal + B + D,
\end{equation}
де всі елементи $m \times m$ матриці $A$ і $m \times m$ кососиметричної матриці $B$ обираються рівномірно випадково з $(-5, 5)$, а усі елементи діагональної матриці $D$ вибираються рівномірно випадково з $(0, 0.3)$ (як наслідок, матриця $M$ додатно визначена), а кожен елемент $q$ обирається рівномірно випадково з $(-500, 0)$. Допустимою множиною є 
\begin{equation}
    C = \left\{ x \in \RR_+^m \middle| x_1 + x_2 + \dots + x_m = m \right\},
\end{equation}
а за початкове наближення береться $x_1 = (1, \dots, 1)$. Для цієї задачі $L = \no{M}$, $\epsilon = 10^{-3}$. \medskip

Допустима множина цієї задачі --- так званий \emph{probability symplex} (з точністю до константи $m$). Для проектування $\vec y$ на нього ми використовували наступний явний
\begin{algorithm}\nothing

    \textbf{Крок 1.} Відсортувати елементи $\vec y$ і зберегти в $\vec u$: $u_1 \ge \dots \ge u_m$. \medskip
        
    \textbf{Крок 2.} Знайти $k = \max j$: $u_j + \frac{1}{j} \left( m - \Sum_{i = 1}^j u_i \right) > 0$. \medskip
        
    \textbf{Крок 3.} Видати вектор з елементами $x_i = \max\{y_i + \lambda, 0\}$, $\lambda = \frac{1}{k} \left( m - \Sum_{i = 1}^k u_i \right)$.
\end{algorithm}

\emph{Цей алгоритм взято із статті} \href{https://stanford.edu/~jduchi/projects/DuchiShSiCh08.pdf}{[J. Duchi, Sh. Shalev-Shwartz, Y. Singer, T. Chandra, 2008]}.

% \section{Третя задача}

% Розглянемо підмножину інтегровних з квадратом на $[0,1]$ функцій з нормою не більше ніж $2$, тобто
% \begin{equation}
%     C \coloneqq \left\{ u \in L^2([0, 1]) : \|u\| = \sqrt{\int_0^1 u^2(x) \diff x} \le 2 \right\}.
% \end{equation}

% Визначимо $F: C \to L^2([0, 1])$ наступним чином:
% \begin{equation}
%     F(u)(t) = \frac{1}{1 + \|u\|^2} \int_0^t u(s) \diff s, \quad \forall u \in L^2([0, 1]), \quad t \in [0, 1].
% \end{equation}

% Це ніщо інше як нормований на $g(u) \coloneqq \frac{1}{1 + \|u\|^2}$ інтегральний оператор Вольтерра, 
% \begin{equation}
%     A(u)(t) = \int_0^t u(s) \diff s.
% \end{equation}

% Як відомо, $A$ --- лінійний, обмежений, і монотонной оператор (вправа 20.12 з [H.~Bauschke, P.~Combettes, Convex analysis and monotone operator theory in hilbert spaces, 2011]), а $\|A\| = \frac{2}{\pi}$. Враховуючи, що $g$ ліпшицева із константою $\frac{16}{25}$ і набуває значень від $\frac{1}{5}$ до $1$, можна показати, що $F$ --- псевдомонотонний ліпшицевий оператор з константою $\frac{82}{\pi}$. \medskip

% У якості початкового наближення візьмемо $u_0(t) = \sin (2 \pi t^2)$. Зауважимо, що вже $A(u_0)(t) = S(t)$, тобто інтегральний сінус (також відомий як інтеграл Френеля), тобто функція не аналітична. Це означає, що нам доведеться якось представляти такі функції. Пропонується користатися координатною системою функцій $\{1, \sin t, \cos t, \sin 2 t, \cos 2 t, \dots, \sin mt, \cos mt\}$. 

\section{Четверта задача}

\begin{equation}
    \begin{aligned}
        F(x) &= F_1(x) + F_2(x), \\
        F_1(x) &= (f_1(x), f_2(x), \dots, f_m(x)), \\
        F_2(x) &= D x + c, \\
        f_i(x) &= x_{i - 1}^2 + x_i^2 + x_{i - 1} x_i + x_i x_{i + 1}, \quad m = 1, 2, \dots, m, \\
        x_0 &= x_{m + 1} = 0,
    \end{aligned}
\end{equation}
де $D$ --- квадратна $m \times m$ матриця з наступними елементами:
\begin{equation}
    d_{i,j} = \begin{cases}
         1, & j = i - 1, \\
         4, & j = i, \\
        -2, & j = i + 1, \\
         0, & \text{інакше},
    \end{cases}
\end{equation}
$c = (-1, -1, \dots, -1)$. Допустимою множиною є $C = \RR_+^m$, а початкова точка $x_1 = (0, 0, \dots, 0)$. \medskip

Для кращого розуміння наведемо матрицю $D$ для кількох перших $m = 3, 4, 5$:
\begin{equation}
    \begin{pmatrix}
        4 & -2 &  0 \\
        1 &  4 & -2 \\
        0 &  1 &  4
    \end{pmatrix}
    \qquad
    \begin{pmatrix}
        4 & -2 &  0 &  0 \\
        1 &  4 & -2 &  0 \\
        0 &  1 &  4 & -2 \\
        0 &  0 &  1 &  4
    \end{pmatrix}
    \qquad
    \begin{pmatrix}
        4 & -2 &  0 &  0 &  0 \\
        1 &  4 & -2 &  0 &  0 \\
        0 &  1 &  4 & -2 &  0 \\
        0 &  0 &  1 &  4 & -2 \\
        0 &  0 &  0 &  1 &  4
    \end{pmatrix}
\end{equation}

\begin{remark}
    Матриця тридіагональна, ми цим скористаємося.
\end{remark}
