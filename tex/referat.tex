\chapter*[Реферат]{РЕФЕРАТ}
Обсяг роботи 56 сторінок, 1 ілюстрація, 9 таблиць, 14 джерел посилань, 1 додаток. \medskip

\MakeUppercase{Варіаційна нерівність, проективний метод, монотонний оператор, екстраградієнтний метод.} \medskip

Об'єктом роботи є процес розв'язування варіаційної нерівності за допомогою власноруч розробленого програмного засобу. Предметом роботи є власноруч розроблений програмний засіб розв'язування варіаційної нерівності. \medskip

Метою роботи є розробка програмного засібу розв'язування варіційної нерівності за допомогою алгоритмів Корпелевич, Tseng'a, Попова та Маліцького---Tam'a, тестування програмного засобу на модельних задачах, оцінка і аналіз результатів. \medskip

Методи розроблення: комп'ютерне моделювання, розробка програмного продукту. Інструменти розроблення: текстовий редактор Sublime Text 3, Jupyter Note\-book, мова програмування Python. \medskip

Результати роботи: розроблено програмний засіб розв'язування варіційної нерівності за допомогою алгоритмів Корпелевич, Tseng'a, Попова та Маліцького---Tam'a, виконано тестування програмного засобу на модельних задачах, проведена оцінка і аналіз результатів 
