\chapter{Результати адаптивних алгоритмів}

\section{Перша задача}

\begin{figure}[H]
    \centering
    \includegraphics[width=.75\textwidth]{img/1/adapt/time.png}
\end{figure}

Та сама інформація у табличці, для зручності:

\begin{table}[H]
	\centering
	\begin{tabular}{|c||c|c|c|c|}\hline
		Розмір задачі & 1000 & 2000 & 5000 & 10000 \\ \hline \hline
		Адапт. Корпелевич & 0.58 & 3.83 & 26.84 & 101.54 \\ \hline
		Кеш. адапт. Корпелевич & 0.22 & 1.30 & 8.63 & 34.52 \\ \hline
		Адапт. Tseng & 0.67 & 4.34 & 29.54 & 120.72 \\ \hline
		Кеш. адапт. Tseng & 0.20 & 1.10 & 7.69 & 30.38 \\ \hline
		Адапт. Попов & 0.35 & 2.12 & 14.40 & 61.09 \\ \hline
		Кеш. адапт. Попов & 0.07 & 0.38 & 2.42 & 10.48 \\ \hline
		Адапт. Маліцький Tam & 0.42 & 2.87 & 18.57 & 80.46 \\ \hline
		Кеш. адапт. Маліцький Tam & 0.09 & 0.45 & 2.74 & 11.82 \\ \hline
	\end{tabular}
	\caption{Час виконання, секунд}
\end{table}


Алгоритма Попова програє своїй неадаптивній версії. Окрім цього, некешовані версії адаптивних алгоритмів явно програють кешованим. Кешовані версії адаптивних алгоритмів Корпелевич і Tseng'a не поступаються кешованим неадаптивним версіям. \medskip

Щодо кількості ітерацій ситуація схожа:

\begin{table}[H]
	\centering
	\begin{tabular}{|c||c|c|c|c|}\hline
		Розмір задачі & 1000 & 2000 & 5000 & 10000 \\ \hline \hline
		Корпелевич = Tseng & 132 & 137 & 144 & 148 \\ \hline
		Попов & 89 & 92 & 96 & 99 \\ \hline \hline
		Адапт. Корпелевич & 125 & 129 & 135 & 139 \\ \hline
		Адапт. Tseng & 125 & 129 & 135 & 139 \\ \hline
		Адапт. Попов & 179 & 185 & 194 & 201 \\ \hline
	\end{tabular}
	\caption{Число ітерацій}
\end{table}


\section{Перша задача із розрідженими матрицями}

\begin{figure}[H]
    \centering
    \includegraphics[width=.75\textwidth]{img/1/sparse/adapt/time.png}
\end{figure}

Та сама інформація у табличці:

\begin{table}[H]
	\centering
	\begin{tabular}{|c||c|c|c|c|}\hline
		Розмір задачі & 50000 & 100000 & 200000 & 500000 \\ \hline \hline
		Адапт. Корпелевич & 0.20 & 1.94 & 4.69 & 12.81 \\ \hline
		Кеш. адапт. Корпелевич & 0.23 & 1.15 & 2.20 & 7.38 \\ \hline
		Адапт. Tseng & 0.28 & 0.93 & 4.72 & 12.61 \\ \hline
		Кеш. адапт. Tseng & 0.10 & 0.49 & 2.14 & 6.69 \\ \hline
		Адапт. Попов & 0.14 & 0.53 & 3.71 & 8.24 \\ \hline
		Кеш. адапт. Попов & 0.06 & 0.21 & 1.23 & 4.31 \\ \hline
	\end{tabular}
	\caption{Час виконання, секунд}
\end{table}


\begin{table}[H]
	\centering
	\begin{tabular}{|c||c|c|c|c|}\hline
		Розмір задачі & 50000 & 100000 & 200000 & 500000 \\ \hline \hline
		Адапт. Корпелевич & 159 & 164 & 169 & 175 \\ \hline
		Кеш. адапт. Корпелевич & 159 & 164 & 169 & 175 \\ \hline
		Адапт. Tseng & 159 & 164 & 169 & 175 \\ \hline
		Кеш. адапт. Tseng & 159 & 164 & 169 & 175 \\ \hline
		Адапт. Попов & 106 & 109 & 112 & 117 \\ \hline
		Кеш. адапт. Попов & 106 & 109 & 112 & 117 \\ \hline
		Адапт. Маліцький Tam & 108 & 111 & 114 & 119 \\ \hline
		Кеш. адапт. Маліцький Tam & 108 & 111 & 114 & 119 \\ \hline
	\end{tabular}
	\caption{Число ітерацій}
\end{table}


Ситуація доволі схожа на попередню, за виключення того що алгоритм Попва тепер не так суттєво програє неадаптивній версії.

\begin{remark}
    З невідомих містичних причин від переходу на розріджені матриці алгоритм Попова починає збігаатися за меншу кількість ітерацій для задачі більшого розміру.
\end{remark}

\section{Друга задача}

\begin{figure}[H]
    \centering
    \includegraphics[width=.75\textwidth]{img/2/adapt/time.png}
\end{figure}

Та сама інформація у табличці:

\begin{table}[H]
	\centering
	\begin{tabular}{|c||c|c|c|c|}\hline
		Розмір задачі & 100 & 200 & 500 & 1000 \\ \hline \hline
		Адапт. Корпелевич & 0.31550021171569825 $\pm$ 0.1138669924879946 & 0.3175917148590088 $\pm$ 0.10412159552178178 & 0.7040012359619141 $\pm$ 0.22415386054565462 & 1.8930572986602783 $\pm$ 0.24094774319954582 \\ \hline
		Кеш. адапт. Корпелевич & 0.1802063465118408 $\pm$ 0.062328725519684706 & 0.22338738441467285 $\pm$ 0.09106931059754596 & 0.5541603565216064 $\pm$ 0.16521676699320012 & 1.2358498573303223 $\pm$ 0.14744469119188727 \\ \hline
		Адапт. Tseng & 0.43882317543029786 $\pm$ 0.13382047347021833 & 0.5867905139923095 $\pm$ 0.16799141193634784 & 1.3596863269805908 $\pm$ 0.16360131485742035 & 3.240447521209717 $\pm$ 0.24623151934728243 \\ \hline
		Кеш. адапт. Tseng & 0.22611584663391113 $\pm$ 0.0712061816394865 & 0.30724201202392576 $\pm$ 0.07657101268229126 & 0.7101064682006836 $\pm$ 0.18943419340029172 & 1.5219688415527344 $\pm$ 0.19355185254638435 \\ \hline
		Адапт. Попов & 0.3998631477355957 $\pm$ 0.07911152470260997 & 0.5405067920684814 $\pm$ 0.16108873929205045 & 0.7611293792724609 $\pm$ 0.27276924260688845 & 2.3248947620391847 $\pm$ 0.33852443429813295 \\ \hline
		Кеш. адапт. Попов & 0.18070569038391113 $\pm$ 0.04384220483555006 & 0.31680846214294434 $\pm$ 0.09004566888338847 & 0.46671152114868164 $\pm$ 0.16160800940835082 & 1.304689645767212 $\pm$ 0.17722127516623767 \\ \hline
	\end{tabular}
	\caption{Час виконання, секунд}
\end{table}


Адаптивні версії суттєво випереджають неадаптивні, причому за числом ітерацій також:

\begin{table}[H]
	\centering
	\begin{tabular}{|c||c|c|c|c|}\hline
		Розмір задачі & 100 & 200 & 500 & 1000 \\ \hline \hline
		Адапт. Корпелевич & 317 $\pm$ 66 & 359 $\pm$ 42 & 410 $\pm$ 34 & 451 $\pm$ 46 \\ \hline
		Кеш. адапт. Корпелевич & 317 $\pm$ 66 & 359 $\pm$ 42 & 410 $\pm$ 34 & 451 $\pm$ 46 \\ \hline
		Адапт. Tseng & 504 $\pm$ 50 & 684 $\pm$ 38 & 872 $\pm$ 73 & 994 $\pm$ 68 \\ \hline
		Кеш. адапт. Tseng & 504 $\pm$ 50 & 684 $\pm$ 38 & 872 $\pm$ 73 & 994 $\pm$ 68 \\ \hline
		Адапт. Попов & 430 $\pm$ 93 & 507 $\pm$ 64 & 551 $\pm$ 48 & 606 $\pm$ 57 \\ \hline
		Кеш. адапт. Попов & 430 $\pm$ 93 & 507 $\pm$ 64 & 551 $\pm$ 48 & 606 $\pm$ 57 \\ \hline
	\end{tabular}
	\caption{Число ітерацій}
\end{table}

