В рамках дипломної роботи були виконані наступні завдання:
\begin{enumerate}[label=--]
    \item розглянуто широкий спектр алгоритмів розв'язання варіаційної нерівності: алгоритми Корпелевич, Tseng'a, Попова, Маліцького---Tam'a;
    \item розроблено програмний засіб розв'язування варіаційної нерівності за допомогою алгоритмів Корпелевич, Tseng'a, Попова, Маліцького---Tam'a;
    \item виконано тестування алгоримтмів на широкому спектрі модельних задач;
    \item проведено детальний аналіз швидкодії і ефективності алгоритмів.
\end{enumerate} \medskip

Зроблено акцент на ефективності програмної реалізації. \medskip

Зокрема, використовується такий прийом як кешування уже обчислених значень оператора і проектора, що дозволяє пришвидшити алгоритм від 2 до 8 разів, залнжно від особливостей задачі й алгоритму. \medskip

Окрім цього, увага приділяється використанню розріджених матриць, що дозволяє на порядки підняти ефективність алгоритмів для задач із розрідженими матрицями (максимальний розмір задачі, яку можна розв'язати за прийнятний час зростає у \emph{сотні} разів). \medskip

Нарешті, згадується можливість кешування матричних розкладів для пришвидшення мат\-рич\-но-век\-тор\-них операцій для відповідних задач. За нашими оцінками, для складних задач які потребують тисяч ітерацій це може принести виграш у часі роботи приблизно ще на один порядок. \medskip

Автор переконаний, що у сучасному світі для конкретної практичної задачі грамотна програмна реалізація є настільки ж важливою, наскійки й оптимальний алгоритм.