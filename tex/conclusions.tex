У роботі розглянуто широкий спектр алгоритмів розв'язання варіаційної нерівності. Проведено детальний аналіз швидкодії і ефективності алгоритмів. \medskip

Акцент зроблено  на ефективності реалізації. Зокрема, використовуються такі прийоми як кешування уже обчислених значень оператора і проектора, що дозволяє пришвидшити алгоритм від 2 до 4 разів. Окрім цього, увага приділяється використанню розріджених матриць, що дозволяє на порядки підняти ефективність алгоритмів для задач із розрідженими матрицями (максимальний розмір задачі, яку можна розв'язати за прийнятний час зростає у \emph{сотні} разів). Нарешті, згадується можливість кешування матричних розкладів для пришвидшення мат\-рич\-но-век\-тор\-них операцій для відповідних задач. За нашими оцінками, для складних задач які потребують тисяч ітерацій це може принести виграш у часі приблизно на один порядок.