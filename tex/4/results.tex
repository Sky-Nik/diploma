\section{Результати}

А цій задачі константа Ліпшиця мені невідома, тому тут наводяться результати лише адаптивних алгоритмів.

\begin{figure}[H]
    \centering
    \includegraphics[width=.75\textwidth]{img/4/adapt/time.png}
\end{figure}

Та сама інформація у табличці:

\begin{table}[H]
	\centering
	\begin{tabular}{|c||c|c|c|c|}\hline
		Розмір задачі & 500 & 1000 & 2000 & 5000 \\ \hline \hline
		Адапт. Корпелевич & 0.15 & 0.29 & 1.55 & 11.03 \\ \hline
		Кеш. адапт. Корпелевич & 0.06 & 0.10 & 0.53 & 3.75 \\ \hline
		Адапт. Tseng & 0.78 & 1.56 & 9.07 & 67.08 \\ \hline
		Кеш. адапт. Tseng & 0.27 & 0.52 & 2.66 & 19.62 \\ \hline
		Адапт. Попов & 0.10 & 0.21 & 1.21 & 9.15 \\ \hline
		Кеш. адапт. Попов & 0.02 & 0.06 & 0.21 & 1.61 \\ \hline
	\end{tabular}
	\caption{Час виконання, секунд}
\end{table}


\begin{remark}
    Наша реалізація приблизно у 100 разів швидша за результати наведені у статті \href{https://arxiv.org/abs/1502.04968v1}{[Yura Malitsky, 2015]}. 
\end{remark}

\begin{table}[H]
	\centering
	\begin{tabular}{|c||c|c|c|c|}\hline
		Розмір задачі & 500 & 1000 & 2000 & 5000 \\ \hline \hline
		Адапт. Корпелевич & 111 & 113 & 116 & 119 \\ \hline
		Адапт. Tseng & 558 & 572 & 587 & 605 \\ \hline
		Адапт. Попов & 87 & 89 & 91 & 94 \\ \hline
	\end{tabular}
	\caption{Число ітерацій}
\end{table}


\section{Розріджені матриці}

\begin{figure}[H]
    \centering
    \includegraphics[width=.75\textwidth]{img/4/sparse/adapt/time.png}
\end{figure}

Та сама інформація у табличці:

\begin{table}[H]
	\centering
	\begin{tabular}{|c||c|c|c|c|}\hline
		Розмір задачі & 20000 & 50000 & 100000 & 200000 \\ \hline \hline
		Адапт. Корпелевич & 0.15 & 1.28 & 2.36 & 9.33 \\ \hline
		Кеш. адапт. Корпелевич & 0.07 & 0.53 & 1.14 & 3.92 \\ \hline
		Адапт. Tseng & 0.91 & 4.90 & 12.94 & 57.32 \\ \hline
		Кеш. адапт. Tseng & 0.47 & 2.38 & 4.02 & 19.24 \\ \hline
		Адапт. Попов & 0.20 & 0.67 & 1.86 & 10.24 \\ \hline
		Кеш. адапт. Попов & 0.13 & 0.14 & 0.41 & 2.21 \\ \hline
	\end{tabular}
	\caption{Час виконання, секунд}
\end{table}


\begin{table}[H]
	\centering
	\begin{tabular}{|c||c|c|c|c|}\hline
		Розмір задачі & 20000 & 50000 & 100000 & 200000 \\ \hline \hline
		Адапт. Корпелевич & 74 & 76 & 77 & 79 \\ \hline
		Кеш. адапт. Корпелевич & 74 & 76 & 77 & 79 \\ \hline
		Адапт. Tseng & 388 & 399 & 408 & 416 \\ \hline
		Кеш. адапт. Tseng & 388 & 399 & 408 & 416 \\ \hline
		Адапт. Попов & 71 & 73 & 74 & 76 \\ \hline
		Кеш. адапт. Попов & 71 & 73 & 74 & 76 \\ \hline
	\end{tabular}
	\caption{Число ітерацій}
\end{table}


\begin{remark}
    Чомусь на цій задачі алгоритм Tseng'а явно просідає. Причини цього явища поки що не з'ясовані.
\end{remark}

\begin{remark}
    З містичних причин від переходу на розріджені матриці усі алгоритми починають збігаатися за меншу кількість ітерацій для задачі більшого розміру.
\end{remark}
