\section{Задача}

\begin{problem}
    \begin{equation}
        \begin{aligned}
            F(x) &= F_1(x) + F_2(x), \\
            F_1(x) &= (f_1(x), f_2(x), \dots, f_m(x)), \\
            F_2(x) &= D x + c, \\
            f_i(x) &= x_{i - 1}^2 + x_i^2 + x_{i - 1} x_i + x_i x_{i + 1}, \quad m = 1, 2, \dots, m, \\
            x_0 &= x_{m + 1} = 0,
        \end{aligned}
    \end{equation}
    де $D$ --- квадратна $m \times m$ матриця з наступними елементами:
    \begin{equation}
        d_{i,j} = \begin{cases}
             1, & j = i - 1, \\
             4, & j = i, \\
            -2, & j = i + 1, \\
             0, & \text{інакше},
        \end{cases}
    \end{equation}
    $c = (-1, -1, \dots, -1)$. Допустимою множиною є $C = \RR_+^m$, а початкова точка $x_1 = (0, 0, \dots, 0)$.
\end{problem}

Для кращого розуміння наведемо матрицю $D$ для кількох перших $m = 3, 4, 5$:
\begin{equation}
    \begin{pmatrix}
        4 & -2 &  0 \\
        1 &  4 & -2 \\
        0 &  1 &  4
    \end{pmatrix}
    \qquad
    \begin{pmatrix}
        4 & -2 &  0 &  0 \\
        1 &  4 & -2 &  0 \\
        0 &  1 &  4 & -2 \\
        0 &  0 &  1 &  4
    \end{pmatrix}
    \qquad
    \begin{pmatrix}
        4 & -2 &  0 &  0 &  0 \\
        1 &  4 & -2 &  0 &  0 \\
        0 &  1 &  4 & -2 &  0 \\
        0 &  0 &  1 &  4 & -2 \\
        0 &  0 &  0 &  1 &  4
    \end{pmatrix}
\end{equation}

\begin{remark}
    Як бачимо, матриця тридіагональна, чим ми скоро скористаємося.
\end{remark}
