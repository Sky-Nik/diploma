Тестування відбувалося на машині із процесором Intel Core i7-8550U 1.99GHz під 64-бітною версією операційної системи Windows 10.

\section{Перша задача}

\subsection{Неадаптивні алгоритми}

\begin{table}[H]
    \centering
    \begin{tabular}{||c||c|c||c|c|c|c||c|c|c|c||c|c|c|c||} \hline \hline
        \multirow{2}{*}{$m$} & \multicolumn{2}{c||}{Корп.} & \multicolumn{2}{c|}{Tseng} & \multicolumn{2}{c||}{кеш. Tseng} & \multicolumn{2}{c|}{Попов} & \multicolumn{2}{c||}{кеш. Попов} & \multicolumn{2}{c|}{Маліц.} & \multicolumn{2}{c||}{кеш. Маліц.} \\ \cline{2-15}
         & час & ітер. & час & ітер. & час & ітер. & час & ітер. & час & ітер. & час & ітер. & час & ітер. \\ \hline \hline
        1000 & 0.10 & 132 & 0.11 & 132 & 0.08 & 132 & 0.05 & 89 & 0.03 & 89 & 0.08 & 91 & 0.03 & 91 \\ \hline
        2000 & 0.58 & 137 & 0.92 & 137 & 0.53 & 137 & 0.36 & 92 & 0.18 & 92 & 0.54 & 94 & 0.19 & 94 \\ \hline
        5000 & 4.08 & 144 & 6.39 & 144 & 4.34 & 144 & 2.85 & 96 & 1.46 & 96 & 4.37 & 98 & 1.53 & 98 \\ \hline
        10000 & 17.43 & 148 & 26.73 & 148 & 17.82 & 148 & 13.14 & 99 & 6.56 & 99 & 18.94 & 101 & 5.85 & 101 \\ \hline
        \hline
    \end{tabular}
    \caption{Результати  неадаптивних алгоритмів для першої задачі}
    \label{tab:1}
\end{table}

Бачимо що алгоритм Корпелевич і кешований алгоритм Tseng'a справді показують майже однакові результати, що є непрямим доказом їхньої еквівалентності. Однакова кількість ітерацій некешованих і кешованих версій усіх алгоритмів слугує непрямим доказом їхньої еквівалентності. Втім, кешовані версії передбачувано виграють (у 1.5--3 рази) у некешованих версій по часу роботи. З точки зору часу роботи найкращі результати показує кешована версія алгоритму Маліцького---Tam'a, з точки зору числа ітерацій --- алгоритм Попова, хоча алгоритм Маліцького---Tam'a майже йому не поступається.

\newpage\subsection{Адаптивні алгоритми}

\begin{table}[H]
    \centering
    \begin{tabular}{||c||c|c|c|c||c|c|c|c||c|c|c|c||c|c|c|c||} \hline \hline
        \multirow{2}{*}{$m$} & \multicolumn{2}{c|}{Корп.} & \multicolumn{2}{c||}{кеш. Корп.} & \multicolumn{2}{c|}{Tseng} & \multicolumn{2}{c||}{кеш. Tseng} & \multicolumn{2}{c|}{Попов} & \multicolumn{2}{c||}{кеш. Попов} & \multicolumn{2}{c|}{Маліц.} & \multicolumn{2}{c||}{кеш. Маліц.} \\ \cline{2-17}
         & час & ітер. & час & ітер. & час & ітер. & час & ітер. & час & ітер. & час & ітер. & час & ітер. & час & ітер. \\ \hline \hline
        1000 & 0.26 & 132 & 0.07 & 132 & 0.24 & 132 & 0.08 & 132 & 0.13 & 89 & 0.04 & 89 & 0.17 & 91 & 0.03 & 91 \\ \hline
        2000 & 1.74 & 137 & 0.57 & 137 & 2.01 & 137 & 0.57 & 137 & 1.09 & 92 & 0.19 & 92 & 1.39 & 94 & 0.24 & 94 \\ \hline
        5000 & 12.66 & 144 & 4.29 & 144 & 15.40 & 144 & 4.24 & 144 & 8.16 & 96 & 1.38 & 96 & 10.25 & 98 & 1.65 & 98 \\ \hline
        10000 & 51.37 & 148 & 17.61 & 148 & 66.98 & 148 & 17.16 & 148 & 34.25 & 99 & 5.85 & 99 & 40.75 & 101 & 5.84 & 101 \\ \hline
        \hline
    \end{tabular}
    \caption{Результати адаптивних алгоритмів для першої задачі}
    \label{tab:1-adapt}
\end{table}

Бачимо що алгоритм Корпелевич і алгоритм Tseng'a справді показують майже однакові результати, що є непрямим доказом їхньої еквівалентності. Однакова кількість ітерацій некешованих і кешованих версій усіх алгоритмів слугує непрямим доказом їхньої еквівалентності. Втім, кешовані версії передбачувано виграють (у 3--7 разів) у некешованих версій по часу роботи. З точки зору часу роботи найкращі і майже еквівалентні результати показують кешовані версії алгоритмів Маліцького---Tam'a і Попова, з точки зору числа ітерацій --- алгоритм Попова, хоча алгоритм Маліцького---Tam'a майже йому не поступається. Задача доволі проста, а тому адаптивні алгоритми не випереджають звичайні.

\newpage\subsection{Розріджені матриці, неадаптивні алгоритми}

Нескладно помітити, що матриця $A$ дуже розріджена, що наводить на ідею скористатися модулем scipy.sparse для ефективної роботи з розрідженими матрицями. Це дозволить нам розв'язувати задачу для значно більших $m$.

\begin{table}[H]
    \centering
    \begin{tabular}{||c||c|c||c|c|c|c||c|c|c|c||c|c|c|c||} \hline \hline
        \multirow{2}{*}{$m$} & \multicolumn{2}{c||}{Корп.} & \multicolumn{2}{c|}{Tseng} & \multicolumn{2}{c||}{кеш. Tseng} & \multicolumn{2}{c|}{Попов} & \multicolumn{2}{c||}{кеш. Попов} & \multicolumn{2}{c|}{Маліц.} & \multicolumn{2}{c||}{кеш. Маліц.} \\ \cline{2-15}
         & час & ітер. & час & ітер. & час & ітер. & час & ітер. & час & ітер. & час & ітер. & час & ітер. \\ \hline \hline
        50000 & 0.15 & 159 & 0.25 & 159 & 0.16 & 159 & 0.10 & 106 & 0.08 & 106 & 0.13 & 108 & 0.08 & 108 \\ \hline
        100000 & 0.51 & 164 & 0.74 & 164 & 0.39 & 164 & 0.37 & 109 & 0.33 & 109 & 0.43 & 111 & 0.30 & 111 \\ \hline
        200000 & 2.32 & 169 & 3.06 & 169 & 2.53 & 169 & 1.56 & 112 & 1.22 & 112 & 2.13 & 114 & 1.41 & 114 \\ \hline
        500000 & 6.21 & 175 & 8.26 & 175 & 6.87 & 175 & 4.14 & 117 & 3.33 & 117 & 5.74 & 119 & 3.78 & 119 \\ \hline
        \hline
    \end{tabular}
    \caption{Результати неадаптивних алгоритмів для першої задачі із розрідженими матрицями}
    \label{tab:1-sparse}
\end{table}

Бачимо що алгоритм Корпелевич і алгоритм Tseng'a справді показують майже однакові результати, що є непрямим доказом їхньої еквівалентності. Однакова кількість ітерацій некешованих і кешованих версій усіх алгоритмів слугує непрямим доказом їхньої еквівалентності. Тут перевага кешування вже не така очевидна, адже ми значно здешевили обчислення оператора $A$, хоча все ще присутня (у 1.5 рази). З точки зору часу роботи найкращі і майже еквівалентні результати показують кешовані версії алгоритмів Маліцького---Tam'a і Попова, з точки зору числа ітерацій --- алгоритм Попова, хоча алгоритм Маліцького---Tam'a майже йому не поступається. Зауважимо, що розміри задачі зросли у 50--100 разів від викорситання розріджених матриць.

\newpage\subsection{Розріджені матриці, адаптивні алгоритми}

\begin{table}[H]
    \centering
    \begin{tabular}{||c||c|c|c|c||c|c|c|c||c|c|c|c||c|c|c|c||} \hline \hline
        \multirow{2}{*}{$m$} & \multicolumn{2}{c|}{Корп.} & \multicolumn{2}{c||}{кеш. Корп.} & \multicolumn{2}{c|}{Tseng} & \multicolumn{2}{c||}{кеш. Tseng} & \multicolumn{2}{c|}{Попов} & \multicolumn{2}{c||}{кеш. Попов} & \multicolumn{2}{c|}{Маліц.} & \multicolumn{2}{c||}{кеш. Маліц.} \\ \cline{2-17}
         & час & ітер. & час & ітер. & час & ітер. & час & ітер. & час & ітер. & час & ітер. & час & ітер. & час & ітер. \\ \hline \hline
        50000 & 0.89 & 159 & 0.61 & 159 & 0.63 & 159 & 0.45 & 159 & 0.53 & 106 & 0.16 & 106 & 0.34 & 108 & 0.15 & 108 \\ \hline
        100000 & 1.53 & 164 & 0.95 & 164 & 1.13 & 164 & 0.54 & 164 & 0.89 & 109 & 0.35 & 109 & 0.75 & 111 & 0.31 & 111 \\ \hline
        200000 & 6.21 & 169 & 3.58 & 169 & 6.18 & 169 & 3.57 & 169 & 4.05 & 112 & 2.00 & 112 & 4.11 & 114 & 2.03 & 114 \\ \hline
        500000 & 16.27 & 175 & 9.29 & 175 & 16.35 & 175 & 9.29 & 175 & 10.96 & 117 & 5.21 & 117 & 11.03 & 119 & 5.22 & 119 \\ \hline
        \hline
    \end{tabular}
    \caption{Результати адаптивних алгоритмів для першої задачі із розрідженими матрицями}
    \label{tab:1-sparse-adapt}
\end{table}

Бачимо що алгоритм Корпелевич і алгоритм Tseng'a справді показують майже однакові результати, що є непрямим доказом їхньої еквівалентності. Однакова кількість ітерацій некешованих і кешованих версій усіх алгоритмів слугує непрямим доказом їхньої еквівалентності. Тут перевага кешування вже не така очевидна, адже ми значно здешевили обчислення оператора $A$, хоча все ще присутня (у 2 рази). З точки зору часу роботи найкращі і майже еквівалентні результати показують кешовані версії алгоритмів Маліцького---Tam'a і Попова, з точки зору числа ітерацій --- алгоритм Попова, хоча алгоритм Маліцького---Tam'a майже йому не поступається. Задача доволі проста, а тому адаптивні алгоритми не випереджають звичайні. Зауважимо, що розміри задачі зросли у 50--100 разів від викорситання розріджених матриць.

\section{Друга задача}

\subsection{Неадаптивні алгоритми}

\begin{table}[H]
    \centering
    \begin{tabular}{||c||c|c||c|c|c|c||c|c|c|c||c|c|c|c||} \hline \hline
        \multirow{2}{*}{$m$} & \multicolumn{2}{c||}{Корп.} & \multicolumn{2}{c|}{Tseng} & \multicolumn{2}{c||}{кеш. Tseng} & \multicolumn{2}{c|}{Попов} & \multicolumn{2}{c||}{кеш. Попов} & \multicolumn{2}{c|}{Маліц.} & \multicolumn{2}{c||}{кеш. Маліц.} \\ \cline{2-15}
         & час & ітер. & час & ітер. & час & ітер. & час & ітер. & час & ітер. & час & ітер. & час & ітер. \\ \hline \hline
        100 & 0.62 & 967 & 0.48 & 967 & 0.40 & 967 & 0.55 & 967 & 0.43 & 967 & 0.47 & 967 & 0.29 & 967 \\ \hline
        100 & 0.71 & 1151 & 0.58 & 1151 & 0.47 & 1151 & 0.68 & 1151 & 0.58 & 1151 & 0.59 & 1151 & 0.36 & 1151 \\ \hline
        100 & 0.63 & 1086 & 0.54 & 1086 & 0.42 & 1086 & 0.62 & 1086 & 0.50 & 1086 & 0.55 & 1086 & 0.34 & 1086 \\ \hline
        200 & 1.41 & 1849 & 1.07 & 1849 & 0.90 & 1849 & 1.39 & 1849 & 1.26 & 1849 & 1.08 & 1849 & 0.72 & 1849 \\ \hline
        200 & 1.30 & 1672 & 1.01 & 1672 & 0.85 & 1672 & 1.30 & 1672 & 1.10 & 1672 & 1.03 & 1672 & 0.69 & 1672 \\ \hline
        200 & 1.35 & 1715 & 1.13 & 1715 & 0.91 & 1715 & 1.32 & 1715 & 1.16 & 1715 & 1.03 & 1715 & 0.71 & 1715 \\ \hline
        500 & 3.67 & 2523 & 2.34 & 2523 & 2.07 & 2523 & 3.63 & 2523 & 3.30 & 2523 & 2.53 & 2523 & 1.88 & 2523 \\ \hline
        500 & 4.33 & 2543 & 2.74 & 2543 & 2.29 & 2543 & 3.77 & 2543 & 3.27 & 2543 & 2.39 & 2543 & 1.94 & 2543 \\ \hline
        500 & 4.07 & 2725 & 2.62 & 2725 & 2.36 & 2725 & 4.10 & 2725 & 3.48 & 2725 & 2.58 & 2725 & 2.16 & 2725 \\ \hline 
        1000 & 10.09 & 3542 & 8.29 & 3542 & 7.23 & 3542 & 9.62 & 3542 & 8.05 & 3542 & 7.95 & 3543 & 5.07 & 3543 \\ \hline
        1000 & 9.18 & 3570 & 7.37 & 3570 & 6.16 & 3570 & 10.57 & 3570 & 8.05 & 3570 & 7.35 & 3570 & 4.75 & 3570 \\ \hline
        1000 & 9.40 & 3471 & 7.55 & 3471 & 6.67 & 3471 & 9.68 & 3471 & 7.76 & 3471 & 7.34 & 3471 & 5.11 & 3471 \\ \hline
        \hline
    \end{tabular}
    \caption{Результати неадаптивних алгоритмів для другої задачі}
    \label{tab:2}
\end{table}

Однакова кількість ітерацій некешованих і кешованих версій усіх алгоритмів слугує непрямим доказом їхньої еквівалентності. Втім, кешовані версії все ще передбачувано виграють (у 1.5 рази) у некешованих версій по часу роботи. З точки зору часу роботи найкращі результати показує кешована версія алгоритму Маліцького---Tam'a. У цій задачі обчислення проекції вже більш складне, %($O(m \log m)$)
тому алгоритм Tseng'a має певну перевагу над алгоритмом Попова. 

\newpage\subsection{Адаптивні алгоритми}

\begin{table}[H]
    \centering
    \begin{tabular}{||c||c|c|c|c||c|c|c|c||c|c|c|c||c|c|c|c||} \hline \hline
        \multirow{2}{*}{$m$} & \multicolumn{2}{c|}{Корп.} & \multicolumn{2}{c||}{кеш. Корп.} & \multicolumn{2}{c|}{Tseng} & \multicolumn{2}{c||}{кеш. Tseng} & \multicolumn{2}{c|}{Попов} & \multicolumn{2}{c||}{кеш. Попов} & \multicolumn{2}{c|}{Маліц.} & \multicolumn{2}{c||}{кеш. Маліц.} \\ \cline{2-17}
         & час & ітер. & час & ітер. & час & ітер. & час & ітер. & час & ітер. & час & ітер. & час & ітер. & час & ітер. \\ \hline \hline
        100 & 1.14 & 967 & 0.64 & 967 & 0.99 & 967 & 0.44 & 967 & 0.97 & 967 & 0.51 & 967 & 0.89 & 967 & 0.35 & 967 \\ \hline
        100 & 1.16 & 1151 & 0.73 & 1151 & 1.10 & 1151 & 0.52 & 1151 & 1.15 & 1151 & 0.58 & 1151 & 1.09 & 1151 & 0.42 & 1151 \\ \hline
        100 & 1.10 & 1086 & 0.69 & 1086 & 1.02 & 1086 & 0.49 & 1086 & 1.08 & 1086 & 0.56 & 1086 & 1.01 & 1086 & 0.41 & 1086 \\ \hline
        200 & 2.30 & 1849 & 1.59 & 1849 & 2.01 & 1849 & 1.10 & 1849 & 2.37 & 1849 & 1.56 & 1849 & 2.33 & 1849 & 0.90 & 1849 \\ \hline
        200 & 2.06 & 1672 & 1.38 & 1672 & 1.74 & 1672 & 1.01 & 1672 & 2.06 & 1672 & 1.21 & 1672 & 1.88 & 1672 & 0.81 & 1672 \\ \hline
        200 & 2.07 & 1715 & 1.46 & 1715 & 1.82 & 1715 & 0.94 & 1715 & 2.05 & 1715 & 1.27 & 1715 & 1.83 & 1715 & 0.86 & 1715 \\ \hline
        500 & 5.30 & 2523 & 3.88 & 2523 & 3.85 & 2523 & 2.32 & 2523 & 4.68 & 2523 & 3.25 & 2523 & 3.71 & 2523 & 2.03 & 2523 \\ \hline
        500 & 5.08 & 2543 & 4.13 & 2543 & 3.88 & 2543 & 2.50 & 2543 & 5.14 & 2543 & 3.29 & 2543 & 4.16 & 2543 & 2.14 & 2543 \\ \hline
        500 & 5.36 & 2725 & 3.99 & 2725 & 3.99 & 2725 & 2.43 & 2725 & 5.26 & 2725 & 3.50 & 2725 & 4.15 & 2725 & 2.33 & 2725 \\ \hline
        1000 & 15.93 & 3542 & 10.31 & 3542 & 11.56 & 3542 & 6.54 & 3542 & 14.59 & 3542 & 8.05 & 3542 & 12.11 & 3543 & 5.45 & 3543 \\ \hline
        1000 & 14.28 & 3570 & 9.68 & 3570 & 11.33 & 3570 & 6.51 & 3570 & 15.09 & 3570 & 7.89 & 3570 & 11.13 & 3570 & 4.89 & 3570 \\ \hline
        1000 & 14.43 & 3471 & 9.42 & 3471 & 11.47 & 3471 & 6.40 & 3471 & 14.79 & 3471 & 8.50 & 3471 & 11.99 & 3471 & 5.37 & 3471 \\ \hline
        \hline
    \end{tabular}
    \caption{Результати адаптивних алгоритмів для другої задачі}
    \label{tab:2-adapt}
\end{table}

Однакова кількість ітерацій некешованих і кешованих версій усіх алгоритмів слугує непрямим доказом їхньої еквівалентності. Втім, кешовані версії передбачувано виграють (у 3--7 разів) у некешованих версій по часу роботи. З точки зору часу роботи найкращі результати показує кешована версія алгоритму Маліцького---Tam'a. У цій задачі обчислення проекції вже більш складне, %($O(m \log m)$)
тому алгоритм Tseng'a має певну перевагу над алгоритмом Попова. Задача доволі проста, а тому адаптивні алгоритми не випереджають звичайні. Можна додати якийсь із матричних розкладів $M$ для пришвидшення множення $M x$.

\section{Третя задача}

\subsection{Адаптивні алгоритми}

\begin{table}[H]
    \centering
    \begin{tabular}{||c|c||c|c|c|c||c|c|c|c||c|c|c|c||c|c|c|c||} \hline \hline
        \multirow{2}{*}{$x_1$} & \multirow{2}{*}{$\epsilon$} & \multicolumn{2}{c|}{Корп.} & \multicolumn{2}{c||}{кеш. Корп.} & \multicolumn{2}{c|}{Tseng} & \multicolumn{2}{c||}{кеш. Tseng} & \multicolumn{2}{c|}{Попов} & \multicolumn{2}{c||}{кеш. Попов} & \multicolumn{2}{c|}{Маліц.} & \multicolumn{2}{c||}{кеш. Маліц.} \\ \cline{3-18}
        & & час & ітер. & час & ітер. & час & ітер. & час & ітер. & час & ітер. & час & ітер. & час & ітер. & час & ітер. \\ \hline \hline

        \hline
    \end{tabular}
    \caption{Результати адаптивних алгоритмів для третьої задачі}
    \label{tab:3-adapt}
\end{table}

З певних причин на цій задачі суттєво просідає алгоритм Tseng'a. Що стосується решти, то тут алгоритми Корпелевич і Попова випереджають алгоритм Маліцького---Tam'a за числом ітерацій. Щоправда, якщо почати рахувати за числом обчислень оператора і проектора, то  алгоритм Маліцького---Tam'a не буде їм поступатися. Розв'язок цієї задачі неєдиний: для деяких випадкових стартових точок алгоритми збігаються до різних розв'язків, що пояснює велику різницю то числу ітерацій між ними.

\section{Четверта задача}

\subsection{Адаптивні алгоритми}

\begin{table}[H]
    \centering
    \begin{tabular}{||c|c||c|c|c|c||c|c|c|c||c|c|c|c||c|c|c|c||} \hline \hline
        \multirow{2}{*}{$m$} & \multirow{2}{*}{$\epsilon$} & \multicolumn{2}{c|}{Корп.} & \multicolumn{2}{c||}{кеш. Корп.} & \multicolumn{2}{c|}{Tseng} & \multicolumn{2}{c||}{кеш. Tseng} & \multicolumn{2}{c|}{Попов} & \multicolumn{2}{c||}{кеш. Попов} & \multicolumn{2}{c|}{Маліц.} & \multicolumn{2}{c||}{кеш. Маліц.} \\ \cline{3-18}
        & & час & ітер. & час & ітер. & час & ітер. & час & ітер. & час & ітер. & час & ітер. & час & ітер. & час & ітер. \\ \hline \hline
        \multirow{2}{*}{500} & $10^{-3}$ & 0.05 & 35 & 0.02 & 35 & 0.25 & 175 & 0.09 & 175 & 0.04 & 34 & 0.01 & 34 & 0.04 & 31 & 0.01 & 31 \\ \cline{2-18}
        & $10^{-6}$ & 0.09 & 65 & 0.03 & 65 & 0.52 & 343 & 0.17 & 343 & 0.08 & 63 & 0.02 & 63 & 0.13 & 60 & 0.03 & 60 \\ \hline
        \multirow{2}{*}{1000} & $10^{-3}$ & 0.09 & 37 & 0.03 & 37 & 0.53 & 183 & 0.17 & 183 & 0.07 & 35 & 0.01 & 35 & 0.07 & 32 & 0.01 & 32 \\ \cline{2-18}
        & $10^{-6}$ & 0.23 & 67 & 0.06 & 67 & 1.16 & 352 & 0.43 & 352 & 0.13 & 65 & 0.03 & 65 & 0.15 & 62 & 0.03 & 62 \\ \hline
        \multirow{2}{*}{2000} & $10^{-3}$ & 0.48 & 38 & 0.20 & 38 & 3.29 & 192 & 0.82 & 192 & 0.46 & 37 & 0.11 & 37 & 0.56 & 34 & 0.08 & 34 \\ \cline{2-18}
        & $10^{-6}$ & 0.85 & 69 & 0.36 & 69 & 5.23 & 360 & 1.49 & 360 & 0.80 & 66 & 0.14 & 66 & 0.89 & 63 & 0.13 & 63 \\ \hline
        \multirow{2}{*}{5000} & $10^{-3}$ & 4.10 & 40 & 1.20 & 40 & 21.45 & 203 & 6.37 & 203 & 3.36 & 39 & 0.69 & 39 & 3.89 & 36 & 0.53 & 36 \\ \cline{2-18}
        & $10^{-6}$ & 5.95 & 71 & 1.98 & 71 & 36.11 & 371 & 10.37 & 371 & 5.64 & 68 & 1.00 & 68 & 6.34 & 65 & 0.93 & 65 \\ \hline
        \hline
    \end{tabular}
    \caption{Результати адаптивних алгоритмів для четвертої задачі}
    \label{tab:4-adapt}
\end{table}

З певних причин на цій задачі просідає алгоритм Tseng'a. Кешовані версії передбачувано виграють (у 3--7 разів) у некешованих версій по часу роботи. Як з точки зору числа ітерацій, так і з точки зору часу роботи найкращим є алгоритм Маліцього---Tam'a, хоча алгоритм Попова і не дуже сильно йому поступається.

\newpage\subsection{Розріджені матриці, адаптивні алгоритми}

Нескладно помітити, що матриця $A$ дуже розріджена, що наводить на ідею скористатися модулем scipy.sparse для ефективної роботи з розрідженими матрицями. Це дозволить нам розв'язувати задачу для значно більших $m$.

\begin{table}[H]
    \centering
    \begin{tabular}{||c|c||c|c|c|c||c|c|c|c||c|c|c|c||c|c|c|c||} \hline \hline
        \multirow{2}{*}{$m$} & \multirow{2}{*}{$\epsilon$} & \multicolumn{2}{c|}{Корп.} & \multicolumn{2}{c||}{кеш. Корп.} & \multicolumn{2}{c|}{Tseng} & \multicolumn{2}{c||}{кеш. Tseng} & \multicolumn{2}{c|}{Попов} & \multicolumn{2}{c||}{кеш. Попов} & \multicolumn{2}{c|}{Маліц.} & \multicolumn{2}{c||}{кеш. Маліц.} \\ \cline{3-18}
        & & час & ітер. & час & ітер. & час & ітер. & час & ітер. & час & ітер. & час & ітер. & час & ітер. & час & ітер. \\ \hline \hline
        \multirow{2}{*}{20000} & $10^{-3}$ & 0.20 & 43 & 0.09 & 43 & 1.12 & 220 & 0.44 & 220 & 0.20 & 42 & 0.06 & 42 & 0.20 & 39 & 0.05 & 39 \\ \cline{2-18}
        & $10^{-6}$ & 0.35 & 74 & 0.15 & 74 & 1.92 & 388 & 0.76 & 388 & 0.34 & 71 & 0.11 & 71 & 0.41 & 68 & 0.12 & 68 \\ \hline
        \multirow{2}{*}{50000} & $10^{-3}$ & 0.83 & 45 & 0.39 & 45 & 4.14 & 231 & 1.68 & 231 & 0.79 & 44 & 0.21 & 44 & 0.80 & 41 & 0.20 & 41 \\ \cline{2-18}
        & $10^{-6}$ & 1.21 & 76 & 0.54 & 76 & 7.87 & 399 & 2.49 & 399 & 1.18 & 73 & 0.28 & 73 & 1.31 & 70 & 0.29 & 70 \\ \hline
        \multirow{2}{*}{100000} & $10^{-3}$ & 2.00 & 47 & 0.68 & 47 & 11.05 & 240 & 3.52 & 240 & 1.80 & 45 & 0.44 & 45 & 2.05 & 42 & 0.37 & 42 \\ \cline{2-18}
        & $10^{-6}$ & 2.70 & 77 & 1.16 & 77 & 18.07 & 408 & 6.91 & 408 & 2.86 & 74 & 0.71 & 74 & 3.38 & 71 & 0.55 & 71 \\ \hline
        \multirow{2}{*}{200000} & $10^{-3}$ & 7.36 & 48 & 3.00 & 48 & 42.02 & 248 & 15.85 & 248 & 7.15 & 47 & 1.99 & 47 & 7.39 & 44 & 1.76 & 44 \\ \cline{2-18}
        & $10^{-6}$ & 12.98 & 79 & 4.94 & 79 & 72.83 & 416 & 25.79 & 416 & 11.79 & 76 & 3.24 & 76 & 12.50 & 73 & 2.92 & 73 \\ \hline
        \hline
    \end{tabular}
    \caption{Результати адаптивних алгоритмів для четвертої задачі із розрідженими матрицями}
    \label{tab:4-sparse-adapt}
\end{table}

З певних причин на цій задачі просідає алгоритм Tseng'a. Кешовані версії передбачувано виграють (у 3--7 разів) у некешованих версій по часу роботи. Як з точки зору числа ітерацій, так і з точки зору часу роботи найкращим є алгоритм Маліцього---Tam'a, хоча алгоритм Попова і не дуже сильно йому поступається. Зауважимо, що розміри задачі зросли у 50--100 разів від викорситання розріджених матриць.

