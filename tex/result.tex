Тестування відбувалося на машині із процесором Intel Core i7-8550U 1.99GHz під 64-бітною версією операційної системи Windows.

\section{Перша задача, неадаптивні алгоритми}

Для усіх алгоритмів у якості початкового наближення ми брали $x_1 = (1, \dots, 1)$, $\epsilon = 10^{-3}$, $\lambda = 0.4$ (константа Ліпшиця цієї задачі дорівнює одиниці: $L = 1$). 

\begin{figure}[H]
    \centering
    \includegraphics[width=\textwidth]{img/1/time.png}
    \caption{Результати неадаптивних алгоритмів на першій задачі}
\end{figure}

І справді, бачимо що алгоритм Корпелевич і кешований алгоритм Tseng'a справді показують майже однакові результати, а також обидві некешовані версії програють кешованим. Та сама інформація у табличці, для зручності:

\begin{table}[H]
	\centering
	\begin{tabular}{|c||c|c|c|c|}\hline
		Розмір задачі & 1000 & 2000 & 5000 & 10000 \\ \hline \hline
		Корпелевич & 0.10 & 0.70 & 5.23 & 21.51 \\ \hline
		Tseng & 0.11 & 1.01 & 7.22 & 30.14 \\ \hline
		Кеш. Tseng & 0.09 & 0.72 & 4.83 & 19.82 \\ \hline
		Попов & 0.06 & 0.54 & 3.51 & 13.24 \\ \hline
		Кеш. Попов & 0.03 & 0.27 & 1.81 & 6.66 \\ \hline
		Маліцький Tam & 0.09 & 0.71 & 5.34 & 20.53 \\ \hline
		Кеш. Маліцький Tam & 0.03 & 0.24 & 1.77 & 6.96 \\ \hline
	\end{tabular}
	\caption{Час виконання, секунд}
\end{table}


\begin{table}[H]
	\centering
	\begin{tabular}{|c||c|c|c|c|}\hline
		Розмір задачі & 1000 & 2000 & 5000 & 10000 \\ \hline \hline
		Корпелевич = Tseng & 132 & 137 & 144 & 148 \\ \hline
		Попов & 89 & 92 & 96 & 99 \\ \hline
	\end{tabular}
	\caption{Число ітерацій}
\end{table}


\section{Перша задача, адаптивні алгоритми}

\begin{figure}[H]
    \centering
    \includegraphics[width=\textwidth]{img/1/adapt/time.png}
    \caption{Результати адаптивних алгоритмів на першій задачі}
\end{figure}

Та сама інформація у табличці, для зручності:

\begin{table}[H]
	\centering
	\begin{tabular}{|c||c|c|c|c|}\hline
		Розмір задачі & 1000 & 2000 & 5000 & 10000 \\ \hline \hline
		Адапт. Корпелевич & 0.58 & 3.83 & 26.84 & 101.54 \\ \hline
		Кеш. адапт. Корпелевич & 0.22 & 1.30 & 8.63 & 34.52 \\ \hline
		Адапт. Tseng & 0.67 & 4.34 & 29.54 & 120.72 \\ \hline
		Кеш. адапт. Tseng & 0.20 & 1.10 & 7.69 & 30.38 \\ \hline
		Адапт. Попов & 0.35 & 2.12 & 14.40 & 61.09 \\ \hline
		Кеш. адапт. Попов & 0.07 & 0.38 & 2.42 & 10.48 \\ \hline
		Адапт. Маліцький Tam & 0.42 & 2.87 & 18.57 & 80.46 \\ \hline
		Кеш. адапт. Маліцький Tam & 0.09 & 0.45 & 2.74 & 11.82 \\ \hline
	\end{tabular}
	\caption{Час виконання, секунд}
\end{table}


Алгоритма Попова програє своїй неадаптивній версії. \medskip

Щодо кількості ітерацій ситуація схожа:

\begin{table}[H]
	\centering
	\begin{tabular}{|c||c|c|c|c|}\hline
		Розмір задачі & 1000 & 2000 & 5000 & 10000 \\ \hline \hline
		Корпелевич = Tseng & 132 & 137 & 144 & 148 \\ \hline
		Попов & 89 & 92 & 96 & 99 \\ \hline \hline
		Адапт. Корпелевич & 125 & 129 & 135 & 139 \\ \hline
		Адапт. Tseng & 125 & 129 & 135 & 139 \\ \hline
		Адапт. Попов & 179 & 185 & 194 & 201 \\ \hline
	\end{tabular}
	\caption{Число ітерацій}
\end{table}


\section{Перша задача із розрідженими матрицями, неадаптивні алгоритми}

Нескладно помітити, що матриця $A$ дуже розріджена, що наводить на ідею скористатися модулем scipy.sparse для ефективної роботи з розрідженими матрицями. Це дозволить нам розв'язувати задачу для значно більших $m$.

\begin{figure}[H]
    \centering
    \includegraphics[width=\textwidth]{img/1/sparse/time.png}
    \caption{Результати неадаптивних алгоритмів на першій задачі із розрідженими матрицями}
\end{figure}

Та сама інформація у табличці, для зручності:

\begin{table}[H]
	\centering
	\begin{tabular}{|c||c|c|c|c|}\hline
		Розмір задачі & 50000 & 100000 & 200000 & 500000 \\ \hline \hline
		Корпелевич & 0.06 & 0.24 & 2.04 & 3.56 \\ \hline
		Tseng & 0.08 & 0.38 & 1.78 & 5.56 \\ \hline
		Кеш. Tseng & 0.07 & 0.21 & 1.35 & 5.22 \\ \hline
		Попов & 0.04 & 0.10 & 1.03 & 2.82 \\ \hline
		Кеш. Попов & 0.03 & 0.13 & 1.13 & 2.73 \\ \hline
	\end{tabular}
	\caption{Час виконання, секунд}
\end{table}


\begin{table}[H]
	\centering
	\begin{tabular}{|c||c|c|c|c|}\hline
		Розмір задачі & 50000 & 100000 & 200000 & 500000 \\ \hline \hline
		Корпелевич & 159 & 164 & 169 & 175 \\ \hline
		Tseng & 159 & 164 & 169 & 175 \\ \hline
		Попов & 106 & 109 & 112 & 117 \\ \hline
		Маліцький Tam & 108 & 111 & 114 & 119 \\ \hline
	\end{tabular}
	\caption{Число ітерацій}
\end{table}


\begin{remark}
    Тут перевага кешування вже не така очевидна, адже ми значно здешевили обчислення оператора $A$, хоча все ще присутня.
\end{remark}

\section{Перша задача із розрідженими матрицями, адаптивні алгоритми}

\begin{figure}[H]
    \centering
    \includegraphics[width=\textwidth]{img/1/sparse/adapt/time.png}
    \caption{Результати адаптивних алгоритмів на першій задачі із розрідженими матрицями}
\end{figure}

Та сама інформація у табличці:

\begin{table}[H]
	\centering
	\begin{tabular}{|c||c|c|c|c|}\hline
		Розмір задачі & 50000 & 100000 & 200000 & 500000 \\ \hline \hline
		Адапт. Корпелевич & 0.20 & 1.94 & 4.69 & 12.81 \\ \hline
		Кеш. адапт. Корпелевич & 0.23 & 1.15 & 2.20 & 7.38 \\ \hline
		Адапт. Tseng & 0.28 & 0.93 & 4.72 & 12.61 \\ \hline
		Кеш. адапт. Tseng & 0.10 & 0.49 & 2.14 & 6.69 \\ \hline
		Адапт. Попов & 0.14 & 0.53 & 3.71 & 8.24 \\ \hline
		Кеш. адапт. Попов & 0.06 & 0.21 & 1.23 & 4.31 \\ \hline
	\end{tabular}
	\caption{Час виконання, секунд}
\end{table}


\begin{table}[H]
	\centering
	\begin{tabular}{|c||c|c|c|c|}\hline
		Розмір задачі & 50000 & 100000 & 200000 & 500000 \\ \hline \hline
		Адапт. Корпелевич & 159 & 164 & 169 & 175 \\ \hline
		Кеш. адапт. Корпелевич & 159 & 164 & 169 & 175 \\ \hline
		Адапт. Tseng & 159 & 164 & 169 & 175 \\ \hline
		Кеш. адапт. Tseng & 159 & 164 & 169 & 175 \\ \hline
		Адапт. Попов & 106 & 109 & 112 & 117 \\ \hline
		Кеш. адапт. Попов & 106 & 109 & 112 & 117 \\ \hline
		Адапт. Маліцький Tam & 108 & 111 & 114 & 119 \\ \hline
		Кеш. адапт. Маліцький Tam & 108 & 111 & 114 & 119 \\ \hline
	\end{tabular}
	\caption{Число ітерацій}
\end{table}


Ситуація доволі схожа на попередню, за виключення того що алгоритм Попва тепер не так суттєво програє неадаптивній версії.

\section{Друга задача, неадаптивні алгоритми}

Для кожного алгоритму і кожного розміру задачі було проведено $5$ запусків (із різними матрицями), у таблиці і на графіку наведені середні значення та середньоквадратичні відхилення.

\begin{figure}[H]
    \centering
    \includegraphics[width=\textwidth]{img/2/time.png}
    \caption{Результати неадаптивних алгоритмів на другій задачі}
\end{figure}

Та сама інформація у табличці, для зручності:

\begin{table}[H]
	\centering
	\begin{tabular}{|c||c|c|c|c|}\hline
		Розмір задачі & 100 & 200 & 500 & 1000 \\ \hline \hline
		Корпелевич & 0.4350947380065918 $\pm$ 0.1913426633302341 & 0.90950608253479 $\pm$ 0.3372043295482586 & 2.248037815093994 $\pm$ 0.6865537238656925 & 7.862243127822876 $\pm$ 0.8350072687434266 \\ \hline
		Tseng & 0.40647144317626954 $\pm$ 0.15133783648646926 & 0.8260695457458496 $\pm$ 0.1772062299787947 & 1.8909038066864015 $\pm$ 0.4069383251247664 & 6.568820381164551 $\pm$ 0.6888329537322256 \\ \hline
		Кеш. Tseng & 0.38817553520202636 $\pm$ 0.11465933017314325 & 0.7749995708465576 $\pm$ 0.1708256909551331 & 1.591792058944702 $\pm$ 0.42990579543045015 & 5.008282089233399 $\pm$ 0.6336524653402142 \\ \hline
		Попов & 0.42557611465454104 $\pm$ 0.17081812322670448 & 1.0770569324493409 $\pm$ 0.2855756813393744 & 2.7644895553588866 $\pm$ 0.7016420097186868 & 7.4928144931793215 $\pm$ 0.46711583329394046 \\ \hline
		Кеш. Попов & 0.36380467414855955 $\pm$ 0.17619206505491763 & 0.6023744583129883 $\pm$ 0.14104575916143772 & 2.428965997695923 $\pm$ 0.6790339589256428 & 6.353133010864258 $\pm$ 0.4615922554022695 \\ \hline
	\end{tabular}
	\caption{Час виконання, секунд}
\end{table}


У цій задачі основна складність все ще у обчисленні оператора $A$, %($O(m^2)$)
хоча обчислення проекції вже більш складне, %($O(m \log m)$)
тому алгоритм Tseng'a має певну перевагу над алгоритмом Попова, який у свою чергу випереджає алгоритм Корпелевич. Щодо кількості ітерацій то усі три алгоритми демонструють практично ідентичні результати.

\begin{table}[H]
	\centering
	\begin{tabular}{|c||c|c|c|c|}\hline
		Розмір задачі & 100 & 200 & 500 & 1000 \\ \hline \hline
		Корпелевич & 1000.00 $\pm$ 0.00 & 1000.00 $\pm$ 0.00 & 1000.00 $\pm$ 0.00 & 1000.00 $\pm$ 0.00 \\ \hline
		Tseng & 1000.00 $\pm$ 0.00 & 1000.00 $\pm$ 0.00 & 1000.00 $\pm$ 0.00 & 1000.00 $\pm$ 0.00 \\ \hline
		Кеш. Tseng & 1000.00 $\pm$ 0.00 & 1000.00 $\pm$ 0.00 & 1000.00 $\pm$ 0.00 & 1000.00 $\pm$ 0.00 \\ \hline
		Попов & 1000.00 $\pm$ 0.00 & 1000.00 $\pm$ 0.00 & 1000.00 $\pm$ 0.00 & 1000.00 $\pm$ 0.00 \\ \hline
		Кеш. Попов & 1000.00 $\pm$ 0.00 & 1000.00 $\pm$ 0.00 & 1000.00 $\pm$ 0.00 & 1000.00 $\pm$ 0.00 \\ \hline
		Маліцький Tam & 1000.00 $\pm$ 0.00 & 1000.00 $\pm$ 0.00 & 1000.00 $\pm$ 0.00 & 1000.00 $\pm$ 0.00 \\ \hline
		Кеш. Маліцький Tam & 1000.00 $\pm$ 0.00 & 1000.00 $\pm$ 0.00 & 1000.00 $\pm$ 0.00 & 1000.00 $\pm$ 0.00 \\ \hline
	\end{tabular}
	\caption{Число ітерацій}
\end{table}


Знову ж таки, кешування дає перевагу на великих задачах, хоча вона вже не у 1.5--2 рази.

\begin{remark}
    Можна додати якийсь із матрчиних розкладів $M$ для пришвидшення множення $M x$. 
\end{remark}

\section{Друга задача, адаптивні алгоритми}

\begin{figure}[H]
    \centering
    \includegraphics[width=\textwidth]{img/2/adapt/time.png}
    \caption{Результати адаптивних алгоритмів на другій задачі}
\end{figure}

Та сама інформація у табличці:

\begin{table}[H]
	\centering
	\begin{tabular}{|c||c|c|c|c|}\hline
		Розмір задачі & 100 & 200 & 500 & 1000 \\ \hline \hline
		Адапт. Корпелевич & 0.31550021171569825 $\pm$ 0.1138669924879946 & 0.3175917148590088 $\pm$ 0.10412159552178178 & 0.7040012359619141 $\pm$ 0.22415386054565462 & 1.8930572986602783 $\pm$ 0.24094774319954582 \\ \hline
		Кеш. адапт. Корпелевич & 0.1802063465118408 $\pm$ 0.062328725519684706 & 0.22338738441467285 $\pm$ 0.09106931059754596 & 0.5541603565216064 $\pm$ 0.16521676699320012 & 1.2358498573303223 $\pm$ 0.14744469119188727 \\ \hline
		Адапт. Tseng & 0.43882317543029786 $\pm$ 0.13382047347021833 & 0.5867905139923095 $\pm$ 0.16799141193634784 & 1.3596863269805908 $\pm$ 0.16360131485742035 & 3.240447521209717 $\pm$ 0.24623151934728243 \\ \hline
		Кеш. адапт. Tseng & 0.22611584663391113 $\pm$ 0.0712061816394865 & 0.30724201202392576 $\pm$ 0.07657101268229126 & 0.7101064682006836 $\pm$ 0.18943419340029172 & 1.5219688415527344 $\pm$ 0.19355185254638435 \\ \hline
		Адапт. Попов & 0.3998631477355957 $\pm$ 0.07911152470260997 & 0.5405067920684814 $\pm$ 0.16108873929205045 & 0.7611293792724609 $\pm$ 0.27276924260688845 & 2.3248947620391847 $\pm$ 0.33852443429813295 \\ \hline
		Кеш. адапт. Попов & 0.18070569038391113 $\pm$ 0.04384220483555006 & 0.31680846214294434 $\pm$ 0.09004566888338847 & 0.46671152114868164 $\pm$ 0.16160800940835082 & 1.304689645767212 $\pm$ 0.17722127516623767 \\ \hline
	\end{tabular}
	\caption{Час виконання, секунд}
\end{table}


Адаптивні версії суттєво випереджають неадаптивні, причому за числом ітерацій також:

\begin{table}[H]
	\centering
	\begin{tabular}{|c||c|c|c|c|}\hline
		Розмір задачі & 100 & 200 & 500 & 1000 \\ \hline \hline
		Адапт. Корпелевич & 317 $\pm$ 66 & 359 $\pm$ 42 & 410 $\pm$ 34 & 451 $\pm$ 46 \\ \hline
		Кеш. адапт. Корпелевич & 317 $\pm$ 66 & 359 $\pm$ 42 & 410 $\pm$ 34 & 451 $\pm$ 46 \\ \hline
		Адапт. Tseng & 504 $\pm$ 50 & 684 $\pm$ 38 & 872 $\pm$ 73 & 994 $\pm$ 68 \\ \hline
		Кеш. адапт. Tseng & 504 $\pm$ 50 & 684 $\pm$ 38 & 872 $\pm$ 73 & 994 $\pm$ 68 \\ \hline
		Адапт. Попов & 430 $\pm$ 93 & 507 $\pm$ 64 & 551 $\pm$ 48 & 606 $\pm$ 57 \\ \hline
		Кеш. адапт. Попов & 430 $\pm$ 93 & 507 $\pm$ 64 & 551 $\pm$ 48 & 606 $\pm$ 57 \\ \hline
	\end{tabular}
	\caption{Число ітерацій}
\end{table}


\section{Четверта задача, адаптивні алгоритми}

А цій задачі константа Ліпшиця мені невідома, тому тут наводяться результати лише адаптивних алгоритмів.

\begin{figure}[H]
    \centering
    \includegraphics[width=\textwidth]{img/4/adapt/time.png}
    \caption{Результати адаптивних алгоритмів на четвертій задачі}
\end{figure}

Та сама інформація у табличці:

\begin{table}[H]
	\centering
	\begin{tabular}{|c||c|c|c|c|}\hline
		Розмір задачі & 500 & 1000 & 2000 & 5000 \\ \hline \hline
		Адапт. Корпелевич & 0.15 & 0.29 & 1.55 & 11.03 \\ \hline
		Кеш. адапт. Корпелевич & 0.06 & 0.10 & 0.53 & 3.75 \\ \hline
		Адапт. Tseng & 0.78 & 1.56 & 9.07 & 67.08 \\ \hline
		Кеш. адапт. Tseng & 0.27 & 0.52 & 2.66 & 19.62 \\ \hline
		Адапт. Попов & 0.10 & 0.21 & 1.21 & 9.15 \\ \hline
		Кеш. адапт. Попов & 0.02 & 0.06 & 0.21 & 1.61 \\ \hline
	\end{tabular}
	\caption{Час виконання, секунд}
\end{table}


\begin{table}[H]
	\centering
	\begin{tabular}{|c||c|c|c|c|}\hline
		Розмір задачі & 500 & 1000 & 2000 & 5000 \\ \hline \hline
		Адапт. Корпелевич & 111 & 113 & 116 & 119 \\ \hline
		Адапт. Tseng & 558 & 572 & 587 & 605 \\ \hline
		Адапт. Попов & 87 & 89 & 91 & 94 \\ \hline
	\end{tabular}
	\caption{Число ітерацій}
\end{table}


\section{Четверта задача  із розрідженими матрицями, адаптивні алгоритми}

\begin{figure}[H]
    \centering
    \includegraphics[width=\textwidth]{img/4/sparse/adapt/time.png}
    \caption{Результати адаптивних алгоритмів на четвертій задачі із розрідженими матрицями}
\end{figure}

Та сама інформація у табличці:

\begin{table}[H]
	\centering
	\begin{tabular}{|c||c|c|c|c|}\hline
		Розмір задачі & 20000 & 50000 & 100000 & 200000 \\ \hline \hline
		Адапт. Корпелевич & 0.15 & 1.28 & 2.36 & 9.33 \\ \hline
		Кеш. адапт. Корпелевич & 0.07 & 0.53 & 1.14 & 3.92 \\ \hline
		Адапт. Tseng & 0.91 & 4.90 & 12.94 & 57.32 \\ \hline
		Кеш. адапт. Tseng & 0.47 & 2.38 & 4.02 & 19.24 \\ \hline
		Адапт. Попов & 0.20 & 0.67 & 1.86 & 10.24 \\ \hline
		Кеш. адапт. Попов & 0.13 & 0.14 & 0.41 & 2.21 \\ \hline
	\end{tabular}
	\caption{Час виконання, секунд}
\end{table}


\begin{table}[H]
	\centering
	\begin{tabular}{|c||c|c|c|c|}\hline
		Розмір задачі & 20000 & 50000 & 100000 & 200000 \\ \hline \hline
		Адапт. Корпелевич & 74 & 76 & 77 & 79 \\ \hline
		Кеш. адапт. Корпелевич & 74 & 76 & 77 & 79 \\ \hline
		Адапт. Tseng & 388 & 399 & 408 & 416 \\ \hline
		Кеш. адапт. Tseng & 388 & 399 & 408 & 416 \\ \hline
		Адапт. Попов & 71 & 73 & 74 & 76 \\ \hline
		Кеш. адапт. Попов & 71 & 73 & 74 & 76 \\ \hline
	\end{tabular}
	\caption{Число ітерацій}
\end{table}

