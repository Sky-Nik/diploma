\section{Результати}

Тестування відбувалося на машині із процесором Intel Core i7-8550U 1.99GHz під 64-бітною версією операційної системи Windows.

\begin{figure}[H]
    \centering
    \includegraphics[width=.75\textwidth]{img/1/time.png}
\end{figure}

І справді, бачимо що алгоритм Корпелевич і кешований алгоритм Tseng'a справді показують майже однакові результати, а також обидві некешовані версії програють кешованим. Та сама інформація у табличці, для зручності:

\begin{table}[H]
	\centering
	\begin{tabular}{|c||c|c|c|c|}\hline
		Розмір задачі & 1000 & 2000 & 5000 & 10000 \\ \hline \hline
		Корпелевич & 0.10 & 0.70 & 5.23 & 21.51 \\ \hline
		Tseng & 0.11 & 1.01 & 7.22 & 30.14 \\ \hline
		Кеш. Tseng & 0.09 & 0.72 & 4.83 & 19.82 \\ \hline
		Попов & 0.06 & 0.54 & 3.51 & 13.24 \\ \hline
		Кеш. Попов & 0.03 & 0.27 & 1.81 & 6.66 \\ \hline
		Маліцький Tam & 0.09 & 0.71 & 5.34 & 20.53 \\ \hline
		Кеш. Маліцький Tam & 0.03 & 0.24 & 1.77 & 6.96 \\ \hline
	\end{tabular}
	\caption{Час виконання, секунд}
\end{table}


\begin{table}[H]
	\centering
	\begin{tabular}{|c||c|c|c|c|}\hline
		Розмір задачі & 1000 & 2000 & 5000 & 10000 \\ \hline \hline
		Корпелевич = Tseng & 132 & 137 & 144 & 148 \\ \hline
		Попов & 89 & 92 & 96 & 99 \\ \hline
	\end{tabular}
	\caption{Число ітерацій}
\end{table}


\begin{remark}
    Наша реалізація приблизно у 50 разів швидша за результати наведені у статті \href{https://arxiv.org/abs/1502.04968v1}{[Yura Malitsky, 2015]}. 
\end{remark}

\section{Розріджені матриці}

Нескладно помітити, що матриця $A$ дуже розріджена, що наводить на ідею скористатися модулем scipy.sparse для ефективної роботи з розрідженими матрицями. Це дозволить нам розв'язувати задачу для значно більших $m$.

\begin{figure}[H]
    \centering
    \includegraphics[width=.75\textwidth]{img/1/sparse/time.png}
\end{figure}

Та сама інформація у табличці, для зручності:

\begin{table}[H]
	\centering
	\begin{tabular}{|c||c|c|c|c|}\hline
		Розмір задачі & 50000 & 100000 & 200000 & 500000 \\ \hline \hline
		Корпелевич & 0.06 & 0.24 & 2.04 & 3.56 \\ \hline
		Tseng & 0.08 & 0.38 & 1.78 & 5.56 \\ \hline
		Кеш. Tseng & 0.07 & 0.21 & 1.35 & 5.22 \\ \hline
		Попов & 0.04 & 0.10 & 1.03 & 2.82 \\ \hline
		Кеш. Попов & 0.03 & 0.13 & 1.13 & 2.73 \\ \hline
	\end{tabular}
	\caption{Час виконання, секунд}
\end{table}


\begin{table}[H]
	\centering
	\begin{tabular}{|c||c|c|c|c|}\hline
		Розмір задачі & 50000 & 100000 & 200000 & 500000 \\ \hline \hline
		Корпелевич & 159 & 164 & 169 & 175 \\ \hline
		Tseng & 159 & 164 & 169 & 175 \\ \hline
		Попов & 106 & 109 & 112 & 117 \\ \hline
		Маліцький Tam & 108 & 111 & 114 & 119 \\ \hline
	\end{tabular}
	\caption{Число ітерацій}
\end{table}


\begin{remark}
    Тут перевага кешування вже не така очевидна, адже ми значно здешевили обчислення оператора $A$, хоча все ще присутня.
\end{remark}
