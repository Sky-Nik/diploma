\chapter{Проста тестова задача}

Для порівняння алгоритмів нам знадобляться тестові задачі різної складності та різних розмірів. У якості такої задачі розглянемо:

\begin{problem}
    Класичний приклад. Допустимою множиною є увесь простір: $C = \RR^m$, а $F(x) = Ax$, де $A$ --- квадратна $m \times m$ матриця, елементи якої визначаються наступним чином:
    \begin{equation}
        a_{i,j} = \begin{cases}
            -1, & j = m - 1 - i > i, \\
            1, & j = m - 1 - i < i, \\
            0, & \text{інакше}.
        \end{cases}
    \end{equation}
    
    \begin{remark}
        Тут і надалі нуметрація рядків/стовпчиків матриць, а також елементів масивів починається з нуля. Якщо у вашій мові програмування нумерація починається з одиниці то у виразах вище замість $m - 1$ має бути $m + 1$.
    \end{remark}
    
    Це визначає матрицю, чия бічна діагональ складається з половини одиниць і половини мінус одиниць, а решта елементів якої нульові. Для наглядності наведемо декілька преших матриць, для $m = 2, 4, 6$:
    \begin{equation}
        \begin{pmatrix}
            0 & -1 \\
            1 & 0
        \end{pmatrix}
        \qquad
        \begin{pmatrix}
            0 & 0 & 0 & -1 \\
            0 & 0 & -1 & 0 \\
            0 & 1 & 0 & 0 \\
            1 & 0 & 0 & 0
        \end{pmatrix}
        \qquad
        \begin{pmatrix}
            0 & 0 & 0 & 0 & 0 & -1 \\
            0 & 0 & 0 & 0 & -1 & 0 \\
            0 & 0 & 0 & -1 & 0 & 0 \\
            0 & 0 & 1 & 0 & 0 & 0 \\
            0 & 1 & 0 & 0 & 0 & 0 \\
            1 & 0 & 0 & 0 & 0 & 0
        \end{pmatrix}
    \end{equation}
    
    Для парних $m$ нульовий вектор є розв'язком відповідної варіаційної нерівності \eqref{eq:variational-inequality}. \medskip
    
    Для усіх алгоритмів у якості початкового наближення ми брали $x_1 = (1, \dots, 1)$, $\epsilon = 10^{-3}$, $\lambda = 0.4$ (константа Ліпшиця цієї задачі дорівнює одиниці: $L = 1$). 
\end{problem}

\begin{remark}
    Для цієї задачі $P_C = \text{Id}$, а тому алгоритми Корпелевич і Tseng'a еквівалентні. Втім, некешована версія алгоритму Tseng'a буде працювати повільніше, що ми скоро і побачимо.
\end{remark}
